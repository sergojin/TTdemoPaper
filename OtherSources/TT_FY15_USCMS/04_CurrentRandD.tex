\section{State of the art and on-going R\&D}

\noindent Motivated by the challenges described above, over the past few years, a number of institutions in CMS have been working on R\&D projects related to the development of fast tracking based on the Associative Memory and tracklet-based approaches.

\noindent The AM device solves the combinatorial problem, inherent to this kind of pattern recognition algorithms, by employing a massively parallel architecture to compare each detector hit to a large number of pre-calculated geometrical patterns simultaneously. Then, the selected patterns are processed using fast FPGAs to perform track fitting. Since each pattern corresponds to a very narrow "road" through the detector, the usual helical fit is much simplified and fast by using a pre-calculated set of parameter values for the center of the road and applying corrections that are a linear function of the actual hit positions in each layer. 

\noindent INFN has been working together with the Atlas FTK team exploring the possibility of using FTK associative memory chips for CMS applications. Development of the simulations for the Associative Memory approach in CMS was started by the Pisa and Lyon groups with significant progress made towards making the machinery work in the standalone mode. Now, together with Padova, they are migrating the tool into CMSSW. Lyon group has been also focusing on the possibility of using a new algorithm, based on Hough's transforms, to replace the conventional SVT-style, linearized track fitting algorithm. 

\noindent Fermilab carried on a focused R\&D program to advance the state-of-the-art of pattern recognition and track reconstruction for fast triggering. Specifically, Fermilab has been developing a new Data Formatting system based on full-mesh ATCA and exploring new Associative Memory VLSI structures based on new technologies both 2-dimensional and 3-dimensional. The 3D approach to Associative Memory implementation is particularly appealing because adding the "third" dimension opens up the possibility for new architectures that would dramatically increase the achievable density of patterns and the pattern recognition power. 

\noindent The long-term goal of this R\&D efforts is to develop all necessary critical technologies to the point where we can ultimately propose them as a viable solution to the CMS L1 tracking trigger problem for HL-LHC. Major progresses have been made recently. INFN has done the initial measurement of AMchip03 latency, Lyon's group had already initial success with the simulation of the Associative Memory and has performed some detailed studies on an algorithm based on Hough's transform. Fermilab has successfully tested the first Data Formatting system components (ATCA motherboard nicknamed Pulsar-IIa, RTM and mezzanine cards) and they all work well, actually better than expected. Fermilab is now ready to test the first Associative Memory 2D prototype chips expected to be delivered at the end of 2013.  Pulsar II hardware design is flexible enough to be used as the main workhorse to build the Vertical Slice Demonstration System.

\subsection{PRAM and its Development}
  
\noindent As mentioned earlier, the CDF SVT-style Associative Memory chip (from now on we will call it PRAM, Pattern Recognition Associative Memory, to emphasis its purpose for HEP) is a departure beyond conventional CAMs. Like conventional CAMs, PRAMs store address patterns and look for matches between incoming hits and those addresses for a given detector layer. At this level, the match is expected to be either exact (Binary CAM) or partial (Ternary CAM) and an array of Match Flags is the typical output. A PRAM has an array of Match Flag Latches which capture and hold the results of the match until reset for the next event. As the hits from the various layers of the detector for the same event arrive, the PRAM is looking for more than simple matches from one candidate address to one or more stored address patterns. The PRAM organizes stored address patterns into roads, which are linked arrays of several stored address patterns from different detector layers. Each stored address pattern in a road is from a different layer in the detector system and these linked arrays represent a path or road that a particle might traverse through the layers of the detector (hence the name "road"). The ultimate goal of the PRAM is to match real particle trajectories to those roads. Like a conventional CAM, a PRAM flags a match when a candidate address matches a stored pattern address for a given detector layer. However, before the PRAM does anything with that match, it must find matches in all (or majority of) the elements (layers) that constitute a road.

\noindent It should be emphasized that compared to commercially available CAMs, such as Network Search Engine, the PRAM has the unique ability to search for correlations among input words received on different clock cycles. This is essential for tracking trigger applications since the input words are the detector hits arriving from different layers at different times. They arrive at the chip without any specific timing correlation. Each pattern has to store each fired layer until the pattern is matched or the event is fully processed. Even in the case of a level-1 trigger application, which is largely synchronous, this feature will still be important. One unique feature of this approach is that the pattern recognition of the event is done as soon as the last hit arrives, which makes the approach a promising candidate for L1 track trigger. However, the requirements for L1 track trigger application will be very different from that for L2, and the system interface of the chip has to be fully redesigned and the performance has to be optimized.

\noindent The PRAM pattern density can be improved by optimizing the design in single-layer chips (2D), using custom cell designs with smaller feature size technology. There is an R\&D effort by INFN using 65 nm technology to improve design for Atlas FTK application for L2 trigger (AMchip05 or 06). INFN has now in hands a 65 nm version prototype, developed for FTK purposes, which could be used for initial testing. INFN AM05 has been submitted recently, and the AM06 will be submitted in the Spring 2014. INFN AM06 for FTK with 128 Kpatterns is expected to become available in 2015.

\noindent  There is also an on-going R\&D effort at Fermilab using both conventional 2D and the emerging 3D technology to design future generation of PRAM chip~\cite{bib:VIP-11},\cite{bib:VIP-12} specifically for the needs of the L1 CMS tracking trigger needs (ProtoVIPRAM series). The first 2D prototype chip (ProtoVIPRAM01) is expected to arrive by the end of 2013.

\subsection{Track Fitting} 

\noindent The traditional CDF SVT/FTK-style track fitting stage can be used to benchmark the performance of this stage. We are exploring other new approaches as well, such as Hough transform algorithm and tracklet-based algorithm. All track fitting algorithms can be implemented in FPGA on the PRMs, therefore they can be studied and compared directly using the same vertical slice demonstration setup. More detailed description of the two new approaches will become available in this section later. As a reference, the traditional SVT/FTK-style track fitting is described below~\cite{bib:FTK-10}.

\noindent For a region of detector sufficiently small, a linear approximation gives helix parameters close to those of full helical fit. In other words, for a road narrow enough that a helical fit can be replaced by a simple linear calculation, each of the 5 helix parameters ($p_i$) can be calculated as the vector product of prestored constants ($a_{ij}$) and the hit coordinates ($x_j$): $p_i = a_{i0} + \sum_{i=1}^{N} a_{ij}x_{j}$  where N is the number of coordinates on the track, one for each SCT layer and two for each pixel layer.  Since there are more than 5 coordinates, there are additional linear equations that correspond to constraint equations, again where the constants are prestored.  There are (N - 5) such equations.  

\noindent This linear approximation gives near offline resolution for regions considerably wider than a single road.  A single set of constants will be used for each sector of the detector. The width of the sector at each silicon layer is the size of a physical detector module.  Per sector, 5( N + 1) constants are needed for the helix parameters, and (N - 5)( N + 1) constants are needed for the constraint equations.  The total number of fit constants (FC) per sector is thus N(N + 1).

\noindent Will add some description here about CDF Gigafitter performance with FPGAs~\cite{bib:Ann-09} etc. also with FTK most recent performance with modern FPGAs.
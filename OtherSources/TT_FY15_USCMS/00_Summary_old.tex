\begin{center}
{\Large\bf Executive summary}
\end{center}

\vspace{0.5cm}

\noindent A key goal of the CMS HL-LHC upgrade is to maintain physics acceptances for all basic objects (leptons, photons, jets and MET) at the Level 1 (L1) Trigger level. As such, the introduction of a L1 Tracking-Trigger is of strategic importance.  Consequently, the design of the Phase-II CMS Tracker must allow for an effective implementation of the tracking trigger.  Since the construction of the Phase-II Tracker will take many years, it's design must be finalized soon.  A silicon-based L1 tracking trigger has never been realized at this scale and thus it is imperative that its feasibility be demonstrated before the design of the Phase-II Tracker is finalized.  Silicon-based Level-2 tracking trigger systems were successfully implemented in the past and are being actively explored at present. Experience with these systems will serve as useful input to the design of the CMS L1 tracking trigger, however the higher occupancies anticipated in HL-LHC operation and the low latencies required at L1 (about 10~$\mu$s) present us with a unique set of challenges.

\noindent Motivated by these challenges, several CMS institutions have carried out a focused R\&D program to advance the state-of-the-art in hardware-based pattern recognition and track reconstruction.  We have attempted to address the issues of occupancy and latency by developing a ``full-mesh'' ATCA data dispatching system, higher density AM chips and new algorithms for hardware-based track finding based on  FPGAs.  The long-term goal of this R\&D effort is to develop these critical technologies to the point where we can ultimately propose them as a viable solution to the problems of HL-LHC L1 track triggering.  Given the progress made by this R\&D program in the last few years, we believe it is now time to take the next important  step and establish a Vertical Slice Demonstration System. This system will comprise a full tracking trigger path and will be used with simulated high-luminosity data to measure trigger latency and efficiency, to study overall system performance and to identify appropriate solutions to possible bottlenecks

\noindent The full-mesh ATCA architecture we are proposing for the CMS L1 tracking trigger permits high bandwidth inter-board communication.  The full-mesh backplane is used to time-multiplex the high volume of incoming data in such a way that I/O demands are manageable at the board and chip level.  The resulting architecture is scalable, flexible and will enable us to provide an early technical demonstration using existing technology. The ATCA architecture will allow us to explore and compare various pattern recognition architectures and algorithms within the same platform.  Given that Advanced Mezzanine Card (AMC) specifications are designed to work with both ATCA and microTCA, the architecture naturally allows for the long-term integration of Tracker DAQ (AMC based) and tracking trigger activities.  

\noindent The proposed architecture and system demonstration has been well received in the Phase-II Tracker Upgrade community and we are now working to better define the concept. In this document we will describe the architecture of a possible L1 tracking trigger for CMS and of the Vertical Slice Demonstration System as an R\&D work plan we propose for the Phase-II upgrade program.  This {\itshape living document} is intended to define the requirements of a tracking trigger and organize efforts leading to the Technical Proposal, the Vertical Slice Demonstration System and the Technical Design Report. The proposed architecture for the system demonstration, however, is not meant to be the final system.  As stated above, it is meant to demonstrate the technical feasibility of a L1 Tracking Trigger with the implementation of an affordable vertical slice which can be designed and built within 2-3 years from now using current state of the art technology. This project will help our community focus the attention on the real issues, compare different possible solutions to the fundamental pattern recognition and track fitting problems, and gain the necessary experience to move, in due time, toward the design of the final system.

\clearpage

\section{Schedule, Milestones, and Resources}
\subsection{Overall Schedule}
The schedule of the proposed program is driven by the current state of the generic R\& D efforts and the CMS desire to build the Vertical Slice Demonstrator on the timescale of early 2016.  In order to meet this aggressive deadline, we have to finalize design of the hardware in 2014 with year 2015 dedicated mainly to the firmware development, setting up the demonstration test-stand and performing the actual testing.  

Currently the prototype layout of the Pulsar-IIb is finalized and we expect to have the board available for testing in early Spring 2014.  Detailed studies and testing require development of the associated software and firmware and based on past experience are expected to take 3-6 months. It is not inconceivable that depending of the results of the testing, another round of prototyping may be required. Final revisions will be made to the board and the final version will be submitted for producton in Fall 2014, followed by the production version testing in late 2014. 

Similarly, prototype of the associative memory chip is expected to be received in early 2014. As outlined above, this prototype chip was intended as a proof of concept. For this purpose the design was kept simple and for this version we intentionally did not include all features needed for L1 applications. Preparation for testing as well as testing itself is expected to take 3-6 months. Engineering work to adjust the chip design to accomodate higher pattern density, faster speed and sparsified readout needed for L1 applications will start as soon as funding is available and will take into consideration results of the protoVipram testing. We expect protoVIPRAM2 to be submitted for productionin Fall 2014 with delivery in late 2014 - early 2015. 

Design of the Pattern Recognition Mezzanine card will proceed in parallel and in close communication with the protoVIPRAM2 development with submission taking place at approximately the same time.  Since production of the card is expected to take less time than for the ASIC, the working mezzanine is expected to be available in Fall 2014. Testing will take place in late 2014 prior to the arrival of protoVIPRAM2.

Early 2015 will be dedicated to the final tests of the hardware components and integration. Setup of the  test stand will take place in Spring 2015.  It is expected that most of the engineering effort will be spent on the firmware development for the Pulsar and  PRM cards. Different versions of firmware will be needed for the Pulsar board to function as DIB and PRB. Integration will take place in Spring-Summer 2015 followed by crate-level testing and measurements of the system performance parameters. 

\subsection{Milestones}
\begin{itemize}
\item Pulsar-IIb prototype version fully tested: FY2014Q2
\item Pulsar-IIc design finalized: FY2014Q3
\item Pulsar-IIc production fully tested: FY2015Q4

\item ProtoVIPRAM2 design dedicated for CMS L1 Track Trigger finalized: FY2015Q1
\item ProtoVIPRAM2 fully tested: FY2015Q2

\item Pattern Recognition Mezzanine card design finalized:  FY2015Q1
\item Pattern Recognition Mezzanine card fully tested:  FY2015Q1

\item All software and firmware complete: FY2015Q3
\item Crate level test-stand setup: FY2015Q3
\item Performance parameters reported: FY2015Q4
\end{itemize}

\subsection{Facilities, Equipment, and Other Resources}
Fermilab facilities will be used for development and testing of the hardware.  Dedicated area on the 14th floor of the Fermilab Wilson Hall has been already allocated for the purposes of testing Pulsar and protoVipram. The same area will be used for setting up the crate-level test stand. Equipment such as computers, power supplies, scopes, logic analyzers etc will be provided by Fermilab PPD electical engineering department. Fermilab electrical engineers, who have extensive experience with the development of modern ASIC's and ATCA based high speed electronics will work on the project. A team of young postdocs will provide simulation, software and testing efforts.  

\clearpage